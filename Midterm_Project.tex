\documentclass[man]{apa6}
\usepackage{lmodern}
\usepackage{amssymb,amsmath}
\usepackage{ifxetex,ifluatex}
\usepackage{fixltx2e} % provides \textsubscript
\ifnum 0\ifxetex 1\fi\ifluatex 1\fi=0 % if pdftex
  \usepackage[T1]{fontenc}
  \usepackage[utf8]{inputenc}
\else % if luatex or xelatex
  \ifxetex
    \usepackage{mathspec}
  \else
    \usepackage{fontspec}
  \fi
  \defaultfontfeatures{Ligatures=TeX,Scale=MatchLowercase}
\fi
% use upquote if available, for straight quotes in verbatim environments
\IfFileExists{upquote.sty}{\usepackage{upquote}}{}
% use microtype if available
\IfFileExists{microtype.sty}{%
\usepackage{microtype}
\UseMicrotypeSet[protrusion]{basicmath} % disable protrusion for tt fonts
}{}
\usepackage{hyperref}
\hypersetup{unicode=true,
            pdftitle={Replication of an Experiment from `The Comparative Value of Survival Processing'},
            pdfauthor={Aira Contreras},
            pdfkeywords={Replication, Adaptive Memory, Survival, Graduate Class},
            pdfborder={0 0 0},
            breaklinks=true}
\urlstyle{same}  % don't use monospace font for urls
\usepackage{graphicx,grffile}
\makeatletter
\def\maxwidth{\ifdim\Gin@nat@width>\linewidth\linewidth\else\Gin@nat@width\fi}
\def\maxheight{\ifdim\Gin@nat@height>\textheight\textheight\else\Gin@nat@height\fi}
\makeatother
% Scale images if necessary, so that they will not overflow the page
% margins by default, and it is still possible to overwrite the defaults
% using explicit options in \includegraphics[width, height, ...]{}
\setkeys{Gin}{width=\maxwidth,height=\maxheight,keepaspectratio}
\IfFileExists{parskip.sty}{%
\usepackage{parskip}
}{% else
\setlength{\parindent}{0pt}
\setlength{\parskip}{6pt plus 2pt minus 1pt}
}
\setlength{\emergencystretch}{3em}  % prevent overfull lines
\providecommand{\tightlist}{%
  \setlength{\itemsep}{0pt}\setlength{\parskip}{0pt}}
\setcounter{secnumdepth}{0}
% Redefines (sub)paragraphs to behave more like sections
\ifx\paragraph\undefined\else
\let\oldparagraph\paragraph
\renewcommand{\paragraph}[1]{\oldparagraph{#1}\mbox{}}
\fi
\ifx\subparagraph\undefined\else
\let\oldsubparagraph\subparagraph
\renewcommand{\subparagraph}[1]{\oldsubparagraph{#1}\mbox{}}
\fi

%%% Use protect on footnotes to avoid problems with footnotes in titles
\let\rmarkdownfootnote\footnote%
\def\footnote{\protect\rmarkdownfootnote}


  \title{Replication of an Experiment from `The Comparative Value of Survival Processing'}
    \author{Aira Contreras\textsuperscript{1}}
    \date{}
  
\shorttitle{Adaptive Memory}
\affiliation{
\vspace{0.5cm}
\textsuperscript{1} Brooklyn College of the City University of New York}
\keywords{Replication, Adaptive Memory, Survival, Graduate Class\newline\indent Word count: X}
\usepackage{csquotes}
\usepackage{upgreek}
\captionsetup{font=singlespacing,justification=justified}

\usepackage{longtable}
\usepackage{lscape}
\usepackage{multirow}
\usepackage{tabularx}
\usepackage[flushleft]{threeparttable}
\usepackage{threeparttablex}

\newenvironment{lltable}{\begin{landscape}\begin{center}\begin{ThreePartTable}}{\end{ThreePartTable}\end{center}\end{landscape}}

\makeatletter
\newcommand\LastLTentrywidth{1em}
\newlength\longtablewidth
\setlength{\longtablewidth}{1in}
\newcommand{\getlongtablewidth}{\begingroup \ifcsname LT@\roman{LT@tables}\endcsname \global\longtablewidth=0pt \renewcommand{\LT@entry}[2]{\global\advance\longtablewidth by ##2\relax\gdef\LastLTentrywidth{##2}}\@nameuse{LT@\roman{LT@tables}} \fi \endgroup}


\DeclareDelayedFloatFlavor{ThreePartTable}{table}
\DeclareDelayedFloatFlavor{lltable}{table}
\DeclareDelayedFloatFlavor*{longtable}{table}
\makeatletter
\renewcommand{\efloat@iwrite}[1]{\immediate\expandafter\protected@write\csname efloat@post#1\endcsname{}}
\makeatother
\usepackage{lineno}

\linenumbers

\authornote{This project has been done for Professor Crumps' Spring 2019 Special Topics in Experimental Psychology Pscy. 7709G.

Correspondence concerning this article should be addressed to Aira Contreras, Not Applicable. E-mail: \href{mailto:aira.contreras@gmail.com}{\nolinkurl{aira.contreras@gmail.com}}}

\abstract{
The experiments in the Nairne, Pandeirada, and Thompson (2008) paper test human memory systems in survival and non-survival conditions in an effort to determine if one yields measurably better recollection from participants. Participants were asked to rate words in conditions that produce excellent retention including conditions that had words related to pleasantness, imagery, and self reference. Previous experiments have suggested that participants show superior memory when words were related to survival conditions. The researchers suggest that this may be a result of fitness advantages accrued in the ancestral past. The goal of this exercise is to replicate on of the experiments presented in the (Nairne et al., 2008) paper and determine if the results were similar or even the same. The analysis will be done via the Rstudio software. I was able to {[}successfully/not successfully{]} replication the experiment.


}

\begin{document}
\maketitle

\hypertarget{methods}{%
\section{Methods}\label{methods}}

We report how we determined our sample size, all data exclusions (if any), all manipulations, and all measures in the study.

\hypertarget{participants}{%
\subsection{Participants}\label{participants}}

\hypertarget{material}{%
\subsection{Material}\label{material}}

\hypertarget{procedure}{%
\subsection{Procedure}\label{procedure}}

\hypertarget{data-analysis}{%
\subsection{Data analysis}\label{data-analysis}}

We used R (Version 3.5.2; R Core Team, 2018) and the R-packages \emph{devtools} (Version 2.0.1; Wickham et al., 2018), \emph{papaja} (Version 0.1.0.9842; Aust \& Barth, 2018), \emph{pwr} (Version 1.2.2; Champely, 2018), \emph{readxl} (Version 1.3.1; Wickham \& Bryan, 2019), and \emph{usethis} (Version 1.4.0; Wickham \& Bryan, 2018) for all our analyses.

\hypertarget{results}{%
\section{Results}\label{results}}

\hypertarget{discussion}{%
\section{Discussion}\label{discussion}}

\newpage

\hypertarget{references}{%
\section{References}\label{references}}

\begingroup
\setlength{\parindent}{-0.5in}
\setlength{\leftskip}{0.5in}

\hypertarget{refs}{}
\leavevmode\hypertarget{ref-R-papaja}{}%
Aust, F., \& Barth, M. (2018). \emph{papaja: Create APA manuscripts with R Markdown}. Retrieved from \url{https://github.com/crsh/papaja}

\leavevmode\hypertarget{ref-R-pwr}{}%
Champely, S. (2018). \emph{Pwr: Basic functions for power analysis}. Retrieved from \url{https://github.com/heliosdrm/pwr}

\leavevmode\hypertarget{ref-nairne2008adaptive}{}%
Nairne, J. S., Pandeirada, J. N., \& Thompson, S. R. (2008). Adaptive memory: The comparative value of survival processing. \emph{Psychological Science}, \emph{19}(2), 176--180.

\leavevmode\hypertarget{ref-R-base}{}%
R Core Team. (2018). \emph{R: A language and environment for statistical computing}. Vienna, Austria: R Foundation for Statistical Computing. Retrieved from \url{https://www.R-project.org/}

\leavevmode\hypertarget{ref-R-usethis}{}%
Wickham, H., \& Bryan, J. (2018). \emph{Usethis: Automate package and project setup}. Retrieved from \url{https://github.com/r-lib/usethis}

\leavevmode\hypertarget{ref-R-readxl}{}%
Wickham, H., \& Bryan, J. (2019). \emph{Readxl: Read excel files}.

\leavevmode\hypertarget{ref-R-devtools}{}%
Wickham, H., Hester, J., \& Chang, W. (2018). \emph{Devtools: Tools to make developing r packages easier}. Retrieved from \url{https://github.com/r-lib/devtools}

\endgroup


\end{document}
